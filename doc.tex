\documentclass[12pt,a4paper]{article}
\usepackage[portuguese]{babel}
\usepackage[utf8]{inputenc}
\usepackage[T1]{fontenc}
\usepackage{makeidx}
\usepackage{url}
 \title{Documentação - Máquinas de Buscas}
 \author{Fernando Barbosa Neto \and Jeferson de Oliveira Batista}
 \date{20 de Setembro de 2015}
\begin{document}
 \maketitle
 \newpage
 
 \section{Introdução}
 {\paragraph{} Este projeto tem como objetivo a implementação de máquinas de buscas. Estas máquinas de buscas serão baseadas em diferentes tipos de dados. }
 {\paragraph{} Para as implementações das máquinas de buscas, serão utilizados modelos booleanos com Tabelas Hash, com tratamentos de colisões que serão detalhados nas próximas seções desta documentação, e Árvore B. }
 {\paragraph{} O projeto está subdividido em dois módulos: um de indexação e outro de busca. Estes serão também explanados nas seções conseguintes. }
 \newpage
 
 \section{Implementação}  
  {\paragraph{} As implementações utilizadas para as máquinas de buscas foram modelos booleanos com tabela hash, utilizando encadeamento, hashing linear e rehashing como métodos de tratamento de colisões, e com árvore B. Tais implementações serão explanadas nas subseções conseguintes.}
  {\paragraph{} Visando uma busca mais rápida e eficiente, foram determinados módulos de indexação e de busca. Estes hão de ser detalhados sob a subseção 2.3 Formatação de Entrada e Saída.}
  
  \subsection{Tabela Hash}
  {\paragraph{} A implementação das funções que são comuns a todos os tipos de hash estão presentes em \emph{encadeamento.c}.
  Isso porque a forma de tratamento das colisões por encadeamento é a única que faz uso de uma lista encadeada nas chaves da tabela hash.
  A função \emph{Hash* criarHash(int tam, int numDocs)} cria uma nova hash, recebendo como parâmetros a quantidade de chaves que a 
  hash possui (tomamos o cuidado de que esse tamanho fosse sempre um número ímpar) e o número de documentos a serem indexados.
  Uma função de espalhamento \emph{int hashing(Hash *hash, char *str)} é utilizada para se calcular em que posição da tabela hash
  a palavra presente em \emph{char *str} deve ser armazenada a princípio, ou seja, caso não haja colisão. }
  {\paragraph{} A função responsável por armazenar os dados contidos na hash em um arquivo de texto é a \emph{void indexarHash(Hash *hash, FILE *file, char **nomes)} 
  onde \emph{file} é o arquivo onde os dados serão escritos e \emph{nomes} contém os nomes dos documentos. }
  {\paragraph{} No módulo de busca, a função que remonta a hash escrita no módulo de indexação, é a \emph{Hash *montarHash(FILE *file, char ***nomes)}, 
  onde \emph{file} é o arquivo de texto onde a hash foi escrita e \emph{nomes} é a variável responsável por armazenas os nomes dos documentos. }
  {\paragraph{} As funções de inserção e busca na hash são específicas para cada forma de tratamento de colisão.}
  
   
   \subsubsection{Tratamento de colisão por encadeamento}
    {\paragraph{} Caso uma mesma posição na tabela hash seja calculada para várias palavras distintas, esse tipo de tratamento apenas armazenada
    tais palavras em uma lista encadeada, no caso da inserção. No caso da busca, as palavras são procuradas em suas respectivas listas encadeadas.}
   
   \subsubsection{Tratamento de colisão por hashing linear}
    {\paragraph{} No caso da hash linear, caso a posição calculada para uma palavra já esteja ocupada, a palavra simplesmente é
    armazenada na próxima posição livre, utilizando-se para isso um incremento circular, isso, na inserção. Durante a busca, começa-se
    a busca pela posição na tabela hash calculada para ela e segue-se a busca, com incremento circular, até que, ou a palavra seja
    encontrada, ou uma posição vaga seja entrada (no caso da palavra não estar presente). }
    
   \subsubsection{Tratamento de colisão por rehashing}
    {\paragraph{} Mecanisnos de inserção e busca semelhantes ao da hash linear são usados aqui, a diferença é que, ao invés do uso
    de incremento circular, há uma nova função de hashing que calcula qual a próxima posição em que se deve tentar inserir a palavra,
    ou qual a próxima posição em que ela deve ser buscada. Caso o tamanho de uma hash que utiliza rehashing não seja um número primo,
    corre-se o risco da função de rehashing repetir a posição calculada, o que poderia levar a um ciclo infinito de tentativas de inserção.}
    
  \subsection{Árvore B}
   {\paragraph{} Na implementação da Árvore B, temos a função \emph{ArvoreB *criarArvoreB(int numDocs)} para que uma nova árvore
   seja criada. O inteiro \emph{numDocs} indica a quantidade de documentos a serem indexados. }
   {\paragraph{} A função principal de inserção na Árvore B é a \emph{ArvoreB *insere\_ArvoreB(ArvoreB *raiz, Termo *t, int numDoc)}, que
   faz uso de várias funções auxiliares para que a palavra presente em \emph{t} seja armazenada. O inteiro \emph{numDocs} indica
   em que documento a palavra foi encontrada. }
   {\paragraph{} Durante a indexação, a principal função utilizada para escrever os dados da Árvore B em um arquivo é a \emph{void indexarArvoreB(ArvoreB *arv, FILE *file, char **nomes)},
   em que \emph{file} é o arquivo de texto onde a Árvore B será escrita e \emph{nomes} armazena os nomes dos documentos a serem
   indexados. }
   {\paragraph{} No módulo de busca, a função responsável por recriar a Árvore B é a \emph{ArvoreB *montarArvoreB(FILE *file, char ***nomes)},
   onde \emph{file} é o arquivo onde os dados da Árvore B foram escritos e \emph{nomes} é a variável responsável por armazenar os nomes
   dos documentos indexados. A busca é feita principalmente através da função \emph{void buscarArvoreB(ArvoreB *arv, char *str, char **nomes)}, 
   onde \emph{str} armazena a palavra a ser buscada e \emph{nomes} os nomes dos documentos. }
   
  \subsection{Formatação de Entrada e Saída}
   {\paragraph{} Há duas subdivisões, como mencionadas anteriormente, que são o Módulo de Indexação e o Módulo de Busca. Ambos os módulos possuem um \emph{Usage} genérico da forma \emph{./trab2 modulo tipoModelo arquivo1 arquivo2}, sendo utilizado em um terminal para a execução do programa. O \emph{modulo} definido no \emph{Usage} deve ser setado para \emph{-i}, no caso de módulo de indexação, ou \emph{-b}, no caso de módulo de busca. Este \emph{modulo} é manuseado no arquivo main.c, de forma a enviar os demais parâmetros para a função adequada, conforme o módulo desejado. Enquanto o \emph{tipoModelo} é o modelo utilizado para criar o índice, podendo ser \emph{E}, utilizando hash e tratamento de colisão por encadeamento, \emph{L} utilizando hash e tratamento de colisão por hashing linear, \emph{R}, utilizando hash e tratamento de colisão por rehashing, e \emph{B}, utilizando árvore B.}
   {\paragraph{} Nas subseções abaixo serão explicadas especificamente cada um dos módulos, isto é, de indexação e de busca.}
   
   \subsubsection{Módulo de Indexação}
    {\paragraph{} O módulo de indexação gesticula os parâmetros nos arquivos index.c e index.h a fim de determinar qual método será criado o índice. }
    {\paragraph{} A função principal para determinar a criação dos índices é a com assinatura \emph{void modulo\_de\_indexacao(char tipoModelo[], char entradaDocs[], char indiceGerado[]);}. \emph{tipoModelo} indica qual modelo de hash ou árvore B será utilizado. \emph{entradaDocs} é o nome do arquivo que possui os nomes dos arquivos a serem indexados. Por sua vez o \emph{indiceGerado} é o nome do arquivo que terá o índice armazenado. }
    {\paragraph{} Dentro da função \emph{modulo\_de\_indexacao}, são abertos os arquivos que contém os arquivos a serem lidos e o arquivo a ser escrito o índice. Para cada arquivo do primeiro, este é aberto e seu conteúdo é armazenado em uma árvore binária. O TAD árvore binária está definida em arvbin.c e arvbin.h. Observe que, através de uma expressão regular, somente as letras, acentuadas ou não, números e hífen são identificados como caracteres válidos de uma palavra. Os caracteres são todos passados para seus formatos em minúsculo, inclusive os acentuados, visto que os acentos não são removidos. Com a árvore binária preenchida com os conteúdos dos arquivos a serem indexados, uma cadeia de \emph{if}/\emph{else} determina qual será o método de indexação segundo o valor passado por \emph{tipoModelo}. }
    {\paragraph{} O arquivo de índice é estruturado de forma que na primeira linha tenha-se o tamanho da hash (ou árvore B) seguido da quantidade de documentos indexado, os quais são imediatamente listados. Após isto, são listadas as palavras da seguinte maneira: na primeira linha há o número da posição na hash (ou árvore B), na segunda há a palavra e as posições nas quais ela foi encontrada e finalizando na terceira linha com uma \emph{flag} \emph{*end*}. }
   
   \subsubsection{Módulo de Busca}
    {\paragraph{} O módulo de busca é tratado majoritariamente pelo arquivo busca.c com sua respectiva biblioteca busca.h.}
    {\paragraph{} A função principal responsável pelas buscas é \emph{void modulo\_de\_busca(char tipoModelo[], char indice[], char arquivoBuscas[]);}, onde \emph{tipoModelo} indica qual modelo de hash ou árvore B será utilizado, \emph{indice} é o nome do arquivo com o índice armazenado e \emph{arquivoBuscas} é o nome do arquivo que contém as \emph{queries}.}
    {\paragraph{} Uma seção de \emph{if}s e \emph{else}s identificam qual modelo de hash ou árvore B está sendo trabalhado. Identificado o modelo, uma expressão regular identifica a primeira palavra do arquivo de buscas. Caso o primeiro caracter não é aspa dupla, a expressão é direcionada para a respectiva função buscadora correspondente ao método utilizado (exemplo: função buscarLinHash para busca em Hash com hashing linear). Caso a expressão começa e termina por aspas duplas, as mesmas são removidas e enviadas para a mesma função do caso anterior. Caso comece mas não termine com aspas duplas, será utilizada a função \emph{void trataBufferAspas(char buffer[]);}, independente do método.}
    {\paragraph{} A função trataBufferAspas realiza, com o apoio da função \emph{Palavra* buscarHashAspas(Hash *hash, char *str);} para buscar determinada palavra em uma hash ou árvore b, o seguinte processo: retira a aspa dupla da primeira palavra, caso esteja junto da palavra, ou pega a próxima palavra para, ao encontrar suas posições com buscarHashAspas, utilizar as posições como base de forma que as palavras seguintes tenho posições que imediatamente sucedem estas posições-base. A palavra que terminar com aspa dupla determina o fim da expressão da \emph{query} (neste caso também é removida a aspa). Terminada a análise da \emph{query}, busca-se quais documentos tiveram continuidade no encontro das palavras. Os documentos com busca bem-sucedida terão seus nomes enviados para stdout. }
   
  
 \newpage
 \section{Análise}	
  {\paragraph{} Para medição e comparação dos tempos, foi utilizada a função \emph{time} do {\it bash} em um computador com a seguinte configuração: Notebook Dell Vostro 3460, com Sistema Operacional Ubuntu 15.04 64-bit, Memória de 5.7 GiB, processador Intel® Core™ i5-3230M CPU @ 2.60GHz x 4 e placa de vídeo GeForce GT 630M/PCIe/SSE2. No terminal foram rodados os seguintes comandos: \emph{time ./trab2 -i tipoModelo entradaDocs indice} para o módulo e indexação e \emph{time ./trab2 -b tipoModelo indice arquivoBuscas} para o módulo de busca, sendo que o tempo a ser avaliado é o de \emph{user}.}
  {\paragraph{} A Tabela 1 disposta a seguir traz um comparativo para o caso de teste analisado. A lista dos documentos a serem indexados e as \emph{queries} podem ser encontrados nos arquivos entradaDocs e arquivoBuscas, respectivamente. Como há pouca diferença entre os tempos calculados, foram realizados ao menos dois testes para cada módulo, a fim de assegurar a precisão dos tempos de processamento.}
  
\begin{table}[!h]
\centering
\caption{Análise dos tempos}
\label{my-label}
\begin{tabular}{l|c|c|}
\cline{2-3}
                                              & \multicolumn{2}{c|}{Módulo} \\ \hline
\multicolumn{1}{|l|}{Método}                  & Indexação      & Busca      \\ \hline
\multicolumn{1}{|l|}{Hash com encadeamento}   & 1.734s              & 4.768s  \\ \hline
\multicolumn{1}{|l|}{Hash com hashing linear} & 1.748s              & 4.780s  \\ \hline
\multicolumn{1}{|l|}{Hash com rehashing}      & 1.810s              & 4.732s  \\ \hline
\multicolumn{1}{|l|}{Árvore B}                & 1.836s              & 4.820s  \\ \hline
\end{tabular}
\end{table}

{\paragraph{} A diferença relativa ao método mais veloz entre os tempos de indexação e de busca entre os extremos, isto é, o com maior tempo de processamento e o com menor, foi de 5,88\% e 1,86\%, respectivamente. }

{\paragraph{} Apesar da proximidade dos resultados, merece destaque o modelo booleano com tabela hash utilizando tratamento de colisão por encadeamento, alcançando o menor tempo de indexação, e o com tabela hash fazendo uso de rehashing para tratamento de colisão, obtendo o menor tempo de busca. Entretanto, a confiabilidade não pode ser de fato assegurada pois os valores são deveras próximos, ora o pior deles pode obter tempo melhor que os demais, assim como o mais veloz pode se tornar o pior em outro momento. Portanto o desempenho a fins de comparação pode ter maior relação com o arranjo da cache do processador, por exemplo, do que o algoritmo em si.}

{\paragraph{} A árvore B, apesar de ter se saído em algumas baterias de teste melhor que as tabelas hash, apresentou o pior desempenho no geral.}

  
 \newpage
 \section{Conclusão}
  {\paragraph{} Todas implementações apresentaram excelente desempenho, tanto em termos de indexação quanto de busca, apesar de uma díficil análise eficaz da velocidade dos métodos consequente da proximidade dos resultados. }
   {\paragraph{} Os métodos apresentaram-se muito parecidos em termos de desenvolvimento, pois possuíam várias funções análogas e funcionamentos semelhantes. As maiores diferenças na etapa de criação do código eram entre a implementação da Árvore B com as Tabelas Hash, apesar de possuírem várias similaridades, especialmente nas funções específicas dos módulos de busca e indexação. }
  {\paragraph{} É possível inferir que, independentemente da implementação escolhida, o usuário estará bem-servido. }
 
 \newpage
 \section{Bibliografia}
  {\paragraph{} ZIVIANI, N. Projeto de Algoritmos, Cengage Learning.}
  {\paragraph{} Aulas ministradas pela Prof\textsuperscript{\b a} Dr\textsuperscript{\b a} Mariella Berger na disciplina de Estrutura de Dados II da Universidade Federal do Espírito Santo entre os dias 26/08/2015 e 16/09/2015. }
 
\end{document}